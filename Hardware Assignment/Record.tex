\documentclass[12pt]{article}
\usepackage{graphicx}
\usepackage[none]{hyphenat}
\usepackage[english]{babel}
\usepackage{caption}
\usepackage[parfill]{parskip}
\usepackage{hyperref}
\usepackage{import}
\usepackage{booktabs}
\usepackage{circuit}%%Importing Packing circiut.sty created for circuit
\def\inputGnumericTable{}
\usepackage{color}                                            
    \usepackage{array}                                            
    \usepackage{longtable}                                        
    \usepackage{calc}                                             
    \usepackage{multirow}                                         
    \usepackage{hhline}                                           
    \usepackage{ifthen}
\usepackage{array}
\usepackage{amsmath}  
\usepackage{circuitikz}
\usepackage{parallel,enumitem}
\usepackage{listings}
\lstset{
language=tex,
frame=single,
breaklines=true
}
 
\begin{document}
	
	\vspace{3cm}
	
	\title{Probability Hardware Assignment}
	\author{\Large Merugu Balavardhan \\ BT22BTECH11010}
	\date{}

\maketitle

\begin{abstract}
	In this assignment we have made a Random number generator using shift registers
\end{abstract}

%\begin{table}[!h]
%\centering
\input{figs/components.tex}

\section*{Procedure}
\begin{enumerate}[]
	\item Connect 555 timer acording to the figure \ref{555_t_c}
	\begin{figure}[h!]
		\includegraphics[width=7cm, height=6cm]{figs/555.png}
		\caption{Connection in 555 timer circuit}
		\label{555_t_c}
	\end{figure}
	
	\item Connect clock signal of D-Flip flops to the Clock output of 555 timer circuit.
	
	\item The next step in the process would be to make the circuitary in such a way that shift registers for using a 4 D-Flip flops (using two 7474 IC's)
	
	\begin{figure}[h!]
		\begin{center}
			
			\includegraphics[width=11cm, height=6cm]{figs/IC7474.png}
			\caption{Connection in 7474 IC}
			\label{7474_IC}
			
		\end{center}
	\end{figure}
	
	\item The next connection is XOR gate (7486 IC) according to the figure \ref{XOR} 
	
	\begin{figure}
		\centering
		\subimport{figs/}
		{circuit1.tex}
		\caption{Connection in XOR gate}
		\label{XOR}
	\end{figure}



\item  A,B,C,D of the decoder (7447 IC) is connected with $Q_0$,$Q_1$,$Q_2$,$Q_3$ respectively as per the figure \ref{7447}

\begin{figure}[!h]
\begin{center}
\resizebox {\columnwidth} {!} {\input{figs/7447.tex}}
\end{center}
\caption{Connection in Decoder gate}
\label{7447}
\end{figure}


\item Final step is to connect the seven segmented display and then connected it with the dceoder (7447 IC) according to the table \ref{table} and the figure \ref{SSD}


\begin{table}[!h]
	\centering
	\input{figs/7447_disp.tex}
	\caption{Connection of seven segmented display with decoder}
	\label{table}
\end{table}

\begin{figure}[!h]
	\begin{center}
		\resizebox {0.3\columnwidth} {!} {
			\input{figs/sevenseg.tex}
		}
	\end{center}
	\caption{Seven segmented display}
	\label{SSD}
\end{figure}


\section*{Output} 
Output as expected is randomly changing numbers as per the figure \ref{output}
\begin{figure}[h]
	\includegraphics[width=\linewidth]{figs/output.jpg}
	\caption{output}
	\label{output}
\end{figure}













\end{enumerate}
\end{document}